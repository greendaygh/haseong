% Options for packages loaded elsewhere
\PassOptionsToPackage{unicode}{hyperref}
\PassOptionsToPackage{hyphens}{url}
%
\documentclass[
]{book}
\usepackage{amsmath,amssymb}
\usepackage{lmodern}
\usepackage{iftex}
\ifPDFTeX
  \usepackage[T1]{fontenc}
  \usepackage[utf8]{inputenc}
  \usepackage{textcomp} % provide euro and other symbols
\else % if luatex or xetex
  \usepackage{unicode-math}
  \defaultfontfeatures{Scale=MatchLowercase}
  \defaultfontfeatures[\rmfamily]{Ligatures=TeX,Scale=1}
  \setmainfont[]{Nanum Myeongjo}
\fi
% Use upquote if available, for straight quotes in verbatim environments
\IfFileExists{upquote.sty}{\usepackage{upquote}}{}
\IfFileExists{microtype.sty}{% use microtype if available
  \usepackage[]{microtype}
  \UseMicrotypeSet[protrusion]{basicmath} % disable protrusion for tt fonts
}{}
\makeatletter
\@ifundefined{KOMAClassName}{% if non-KOMA class
  \IfFileExists{parskip.sty}{%
    \usepackage{parskip}
  }{% else
    \setlength{\parindent}{0pt}
    \setlength{\parskip}{6pt plus 2pt minus 1pt}}
}{% if KOMA class
  \KOMAoptions{parskip=half}}
\makeatother
\usepackage{xcolor}
\usepackage{graphicx}
\makeatletter
\def\maxwidth{\ifdim\Gin@nat@width>\linewidth\linewidth\else\Gin@nat@width\fi}
\def\maxheight{\ifdim\Gin@nat@height>\textheight\textheight\else\Gin@nat@height\fi}
\makeatother
% Scale images if necessary, so that they will not overflow the page
% margins by default, and it is still possible to overwrite the defaults
% using explicit options in \includegraphics[width, height, ...]{}
\setkeys{Gin}{width=\maxwidth,height=\maxheight,keepaspectratio}
% Set default figure placement to htbp
\makeatletter
\def\fps@figure{htbp}
\makeatother
\setlength{\emergencystretch}{3em} % prevent overfull lines
\providecommand{\tightlist}{%
  \setlength{\itemsep}{0pt}\setlength{\parskip}{0pt}}
\setcounter{secnumdepth}{-\maxdimen} % remove section numbering
\ifLuaTeX
  \usepackage{selnolig}  % disable illegal ligatures
\fi
\IfFileExists{bookmark.sty}{\usepackage{bookmark}}{\usepackage{hyperref}}
\IfFileExists{xurl.sty}{\usepackage{xurl}}{} % add URL line breaks if available
\urlstyle{same} % disable monospaced font for URLs
\hypersetup{
  hidelinks,
  pdfcreator={LaTeX via pandoc}}

\author{}
\date{\vspace{-2.5em}}

\begin{document}
\frontmatter

\mainmatter
\hypertarget{intro}{%
\chapter{Intro}\label{intro}}

\begin{itemize}
\tightlist
\item
  김하성 (Haseong Kim)
\item
  한국생명공학연구원 합성생물학연구소 책임연구원
\item
  과학기술연합대학원대학교 생명공학과 부교수
\end{itemize}

\hypertarget{lectures}{%
\chapter{Lectures}\label{lectures}}

\begin{itemize}
\tightlist
\item
  {[}2021년 2학기 UST 대학원 전공강의{]} {[}현장연구A{]}
\item
  {[}2021년 KRIBB 연구데이터 분석과정{]}
  \href{https://greendaygh.github.io/kribbr2021/}{R을 이용한 생명데이터
  분석}
\item
  {[}2021년 1학기 UST 대학원 전공강의{]} {[}현장연구E{]}
\item
  {[}2021년 1학기 UST 대학원 일반강의{]}
  \href{https://greendaygh.github.io/Rprog2021/}{데이터 사이언스를 위한
  R 프로그래밍}
\item
  {[}2020년 KRIBB 연구데이터 분석과정{]}
  \href{https://greendaygh.github.io/KRIBBR2020/}{R을 이용한 생명데이터
  분석}
\item
  {[}2020년 2학기 UST 대학원 일반강의{]}
  \href{https://greendaygh.github.io/Rstat2020/}{R로 배우는 기초통계}
\item
  {[}2020년 1학기 UST 대학원 일반강의{]}
  \href{https://greendaygh.github.io/Rprog2020/}{R 프로그래밍과
  데이터분석}
\item
  {[}2019년 12월 생물공학회 교육 워크샵{]}
  \href{https://github.com/greendaygh/bioengml}{기계학습 x 생물공학}
\item
  {[}2019년 2학기 UST 대학원 일반강의{]}
  \href{https://greendaygh.github.io/Rstat2019/}{R로 배우는 기초통계}
\item
  {[}2019년 1학기 UST 대학원 일반강의{]}
  \href{https://greendaygh.github.io/Rprog2019/}{R 프로그래밍과
  데이터분석}
\end{itemize}

\hypertarget{media}{%
\chapter{Media}\label{media}}

\begin{itemize}
\tightlist
\item
  {[}2021-10-12{]} {[}바이오인와치{]}
  \href{https://www.bioin.or.kr/board.do?num=311389\&cmd=view\&bid=issue}{제조업을
  혁신할 자연지향설계(Nature Co-Design) 시대 도래}
\item
  {[}2021-04-31{]} {[}한국과총{]}
  \href{http://ebook.kofst.or.kr/book/202104/\#page=71}{인류 번영을 위한
  혁신기술, 합성생물학}
\item
  {[}2021-02-24{]} {[}한국경제, 유료{]}
  \href{https://www.hankyung.com/it/article/202102096254i}{생물학의
  산업적 활용을 촉진하는 바이오파운드리 구축 필요}
\item
  {[}2020-12-17{]} {[}전자신문 외 15건{]}
  \href{https://www.fnnews.com/news/202012161243526206}{AI 만난 미생물
  바이오센서, 유해물질 95\% 식별}
\item
  {[}2020-7-30{]} {[}한국생물공학회{]}
  \href{http://www.btnews.or.kr/bbs/board.php?bo_table=bt_news\&wr_id=342}{딥러닝(CNN)기반
  DNA 서열분석 가이드}
\item
  {[}2020-5-1{]} {[}바이오리소스 인사이트 Vol5{]}
  \href{https://www.kobis.re.kr/go_document?page_name=publication_bio_insight}{합성생물학과
  생명연구자원}
\item
  {[}2020-2-3{]} {[}바이오인에세이 감수{]}
  \href{https://www.bioin.or.kr/board.do?num=304334\&cmd=view\&bid=essay}{2020
  바이오미래기유망기술의 이야기-바이오파운드리}
\item
  {[}2020-2-11{]} {[}바이오인 기술 추천 및 기초자료 작성{]}
  \href{https://www.bioin.or.kr/board.do?num=293995\&cmd=view\&bid=essay\&cPage=1\&cate1=all\&cate2=all2}{2020년
  10대 바이오 미래유망기술 바이오파운드리}
\item
  {[}2019-12-31{]} {[}바이오인에세이 감수{]}
  \href{https://www.bioin.or.kr/board.do?num=292912\&cmd=view\&bid=essay}{2019
  바이오미래기유망기술의 이야기-유전자회로 공정 예측기술}
\item
  {[}2019-12-04{]} {[}바이오인프로{]}
  \href{https://www.bioin.or.kr/board.do?num=292266\&cmd=view\&bid=report\&cPage=1\&cate1=all\&cate2=all2}{바이오파운드리
  동향}
\item
  {[}2019-09-17{]} {[}한국경제{]}
  \href{https://www.hankyung.com/it/article/2019091732761}{자동화 로봇
  기술 덕분에 10년 걸리던 표현형 데이터 수일 內 확보}
\item
  {[}2016-09-20{]} {[}보도자료{]}
  \href{https://www.bioin.or.kr/board.do?num=263870\&cmd=view\&bid=research}{저비용·고효율
  맞춤형 미생물 검색 형광 플랫폼 개발}
\item
  {[}2015-07-27{]} {[}바이오인프로{]}
  \href{https://www.bioin.or.kr/board.do?num=253760\&cmd=view\&bid=report}{합성생물학의
  최신 연구동향}
\item
  {[}2013-02-03{]} {[}지능형합성생물학연구단{]}
  \href{http://www.syntheticbiology.or.kr/board.php?db=sub0302\&no=44\&c=view\&page=8}{합성생물학
  부품 표준화 및 설계 정형화에 관한 연구 동향}
\end{itemize}

\hypertarget{selected-publications}{%
\chapter{Selected publications}\label{selected-publications}}

\href{https://oak.kribb.re.kr/researcher-profile?ep=137\&type=all\&sort_by=dc.date.issued_dt\&order=DESC}{Full
paper list}

\textbf{Haseong Kim }, Wonjae Seong, Eugene Rha, Hyewon Lee, Seong Keun
Kim, Kil Koang Kwon, K H Park, Dae-Hee Lee, Seung Goo Lee (2020)
``Machine learning linked evolutionary biosensor array for highly
sensitive and specific molecular identification'' Biosens Bioelectron.
170(0) : 112670-112670.

\textbf{H. Kim } ``AI, big data, and robots for the evolution of
biotechnology'' GENOM INFORMAT. 17(4) : e44-e44, 2019

SK Kim*, \textbf{H. Kim*}, WC Ahn, KH Park, E Woo, DH Lee, and SG Lee,
Efficient Transcriptional Gene Repression by Type V-A CRISPR-Cpf1 from
Eubacterium eligens, 10.1021/acssynbio.6b00368, 2017

\textbf{H. Kim*}, E. Rha*, W. Seong, SJ Yeom, DH Lee, S G Lee, A
cell-cell communication-based screening system for novel microbes with
target enzyme activities, ACS Synthetic Biology, DOI:
10.1021/acssynbio.5b00287, 2016

\textbf{H. Kim}, KK Kwon, W. Seong, SG Lee, Multi-Enzyme Screening Using
a High-throughput Genetic Enzyme Screening System, The Journal of
Visualized Experiments, (114), e54059, 2016

Y J Kim*, \textbf{H. Kim*}, S H Kim, E Rha, S-L Choi, S J Yeom, H S Kim,
S G Lee, Improved metagenome screening efficiency by random insertion of
T7 promoters J BIOTECHNOL, 230(0) : 47-53, 2016

SL Choi, E Rha, SJ Lee, \textbf{H. Kim}, G Kwon, YS Jeong, YH Rhee, JJ
Song, HS Kim, SG Lee, Towards a Generalized and High-throughput Enzyme
Screening System Based on Artificial Genetic Circuits, ACS Synthetic
Biology, 3(3), 2014

\textbf{H. Kim}, T. Park, and E. Gelenbe, Identifying Disease Candidate
Genes via Large-scale Gene Network Analysis, International Journal of
Data Mining and Bioinformatics, 10(2), 2014

\textbf{H. Kim}, Modelling and analysis of gene regulatory networks
based on the G-network, International Journal of Advanced Intelligence
Paradigms, 6(1), 2014

\backmatter
\end{document}
